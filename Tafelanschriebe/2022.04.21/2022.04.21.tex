\subsection{Stirlingformel}
$\Omega(\log n!)$\\

\begin{align*}
    &n! \sim \sqrt{2\pi n}(\frac{n}{e})^n &&\text{Stirling Formel}\\
    &\Rightarrow \log n! \sim \log(\sqrt{2\pi n}(\frac{n}{e})^n) && \log (x + y) = \log x + \log y\\
    &= \underbrace{\frac{1}{2}\log(2\pi)}_{O(1)} + \underbrace{\log n}_{O(\log n)} + n \log n \underbrace{- n}_{O(n)} && \log 1 = 0\\
    &\sim n \log n
\end{align*}


\subsection{Herleitung vom Wallisschen Produkt}
\underline{\textbf{Beweis:}} Übungen
\renewcommand{\intpihalbe}{\int_{0}^{\frac{\pi}{2}}}
\begin{align*}
    &\int_{0}^{\frac{\pi}{2}} \sin^{n} x\ dx = \frac{n-1}{n} \int_{0}^{\frac{\pi}{2}} \sin^{n -2} x \ dx && (n \geq 2)\\
    &\intpihalbe \sin^{2n} x \ dx = \left[\Pi_{i=1}^{n} \frac{2i -1}{2i}\right] \cdot \frac{\pi}{2}\\
    &\intpihalbe \sin^{2n+1} x \ dx = \left[\Pi_{i=1}^{n} \frac{2i -1}{2i}\right] \underbrace{\intpihalbe \sin x \ dx}_{1} = \Pi_{i=1}^{n} \frac{2i -1}{2i}
\end{align*}
\begin{align*}
    &1 \geq \frac{\intpihalbe \sin^{2n+1} x \ dx}{\intpihalbe \sin^{2n} x \ dx} \text{ weil } \sin^{2n+1} x \leq \sin^{2n} x \ \forall x, n \text{ da } \sin x \in [0, 1] \text{ für } x \in [0, \frac{\pi}{2}]\\
    &\geq \frac{\intpihalbe \sin^{2n+2} x \ dx}{\intpihalbe \sin^{2n} x \ dx} = \frac{\frac{\pi}{2} \cdot \Pi_{i = 1}^{n+1} \frac{2i-1}{2i}}{\frac{\pi}{2} \cdot \Pi_{i = 1}^{n} \frac{2i-1}{2i}} = \frac{2n +1}{2n+2} \underset{n \rightarrow \infty}{\longrightarrow} 1\\
    &\Rightarrow \lim_{n \rightarrow \infty} \frac{\intpihalbe \sin^{2n+1} x \ dx}{\intpihalbe \sin^{2n} x \ dx} = 1\\
    &\Rightarrow 1 = \lim_{n \rightarrow \infty} \frac{\Pi_{i = 1}^{n} \frac{2}{2i+1}}{\frac{\pi}{2} \Pi_{i = 1}^{n} \frac{2i-1}{2i}} = \frac{2}{\pi} \lim_{n \rightarrow \infty} \Pi_{i = 1}^{n} \frac{4i^2}{4i^2 -1} && \square
\end{align*}
\newpage
\subsection{Korollar zur Sterlingformel}

\renewcommand*{\rectangled}[1]{\tikz[baseline=(char.base)]{
  \node[shape=rectangle,draw,inner sep=2pt, red] (char) {\textcolor{black}{#1}};}}
\newcommand{\pihalbe}{\frac{\pi}{2}}
\newcommand{\limmesngegenunendlich}{\lim_{n \rightarrow \infty}}
\newcommand{\pivoneinsbisn}{\Pi_{i = 1}^{n}}
\underline{\textbf{Beweis:}}
\begin{align*}
    \text{Wallis } \Rightarrow \frac{\pi}{2} &= \limmesngegenunendlich \pivoneinsbisn \frac{4i^2}{4i^2-1} = \limmesngegenunendlich \pivoneinsbisn\frac{2i}{2i+1} \frac{2i}{2i-1}\\
    &= \limmesngegenunendlich \frac{1}{2n+1} \pivoneinsbisn\left(\frac{2i}{2i-1} \right)^2 = \limmesngegenunendlich \frac{1}{2n}\pivoneinsbisn \left(\frac{2i}{2i-1} \right)^2\\
\end{align*}
\begin{align*}
    \Rightarrow \rectangled{$\sqrt{\pi n}$} \sim \pivoneinsbisn \frac{2i}{2i-1} = \rectangled{$\frac{\pivoneinsbisn 2i}{\pivoneinsbisn 2i-1}$}\\
    \rectangled{$\pivoneinsbisn 2i = 2^n (n!)$}\\
    \pivoneinsbisn 2i-1 = \frac{\pivoneinsbisn i}{\pivoneinsbisn 2i} = \rectangled{$\frac{(2n)!}{2^n n!}$} && \square
\end{align*}

\subsection{Sterling}
\begin{align*}
    n! \sim \sqrt{2\pi n} (\frac{n}{e})^n
\end{align*}
\underline{\textbf{Beweis:}}\\
Definiere $a_n = \frac{\sqrt{n}\left(\frac{n}{e}\right)^n}{n!}$ und $b_n = a_n \cdot \exp(\frac{1}{12n}) \geq a_n$
\begin{align*}
    \log a_n = \frac{1}{2} \cdot \log n - n + \log \frac{n^n}{n!} = \frac{1}{2} \cdot \log n - n + \sum_{i = 1}^n \log \frac{n}{i}
\end{align*}
\begin{align*}
    \log b_n = \log a_n + \frac{1}{12n} = \frac{1}{2} \log n - n + \sum_{i = 1}^n \log \frac{n}{i} + \frac{1}{12n}
\end{align*}
\begin{align*}
    \log a_{n+1} - \log a_n &= \frac{1}{2} \log (n+1) - (n+1) + \sum_{i =1}^{n+\not1} \log \frac{n+i}{i}\\
    &-\left( \frac{1}{2} \log n - n + \sum_{i = 1}^{n} \log \frac{n}{i}\right)\\
    &= \frac{1}{2} \log \frac{n+1}{n} - 1 + n \log \frac{n+1}{n} = \left(n + \frac{1}{2}\right) \log \frac{n+1}{n} - 1
\end{align*}
\begin{align*}
    \log b_{n+1} - \log b_n = \log a_{n+1} - \log a_n + \frac{1}{12n} - \frac{1}{12(n+1)} = \left( \frac{1}{2} + n\right) \log \frac{n+1}{n} - 1 - \frac{1}{12n(n+1)}
\end{align*}
Hilfsfunktionen:
\begin{align*}
    \phi(x) = \frac{1}{2} \log \frac{1+x}{1-x} - x && \psi(x) = \phi(x) - \frac{x^3}{3(1-x^2)}\\
    \phi(0) = 0 && \psi(0) = 0
\end{align*}
\begin{align*}
    \phi'(x) = \frac{x^2}{1 - x^2} \geq 0 \quad (0 \leq x < 1) && \psi'(x) = - \frac{x^4}{6(1-x^2)^2} \leq 0 \quad (0 \leq x < 1)\\
    \Rightarrow \phi(x) \geq 0 \quad \text{für } 0 \leq x \leq 1 && \psi(x) \leq 0 \quad \text{für } 0 \leq x \leq 1
\end{align*}
\underline{\textbf{Einsetzen:}}
\newcommand{\einsdurchzweinpluseins}{\frac{1}{2n+1}}
\begin{align*}
    x = \frac{1}{2n+1}
\end{align*}



Anmerkung:
\begin{align*}
    &\Rightarrow 0 \leq \frac{1}{2} \log \frac{1 +x}{1-x} - x \leq \frac{x^3}{3(1- x^2)}\\
    &0 \leq \frac{1}{2} \log \frac{1 + \einsdurchzweinpluseins}{1 - \einsdurchzweinpluseins} - \einsdurchzweinpluseins \leq \frac{\left(\einsdurchzweinpluseins\right)^3}{3\left(1 - (\einsdurchzweinpluseins)^2\right)}\\
    &= \frac{1}{2} \log \frac{n+1}{n} - \einsdurchzweinpluseins \leq \frac{1}{12n(n+1)(2n+1)}\\
    &\Rightarrow 0 \leq (n + \frac{1}{2}) \log \frac{n+1}{n} - 1 \leq \frac{1}{12n(n+1)}
\end{align*}
Daraus folgt
\begin{align*}
    &\Rightarrow \log a_{n+1} \geq \log a_n\\
    &\log b{n+1} \leq b_n\\
    &\Rightarrow a_{n+1} \geq a_n\\
    &b_{n+1} \leq b_n
\end{align*}
Also wissen wir:
\begin{align*}
    &a_n \leq a_{n+1} \leq b_{n+1} \leq b_n \text{ und }\\ &\limmesngegenunendlich \frac{a_n}{b_n} = \limmesngegenunendlich \exp\left( - \frac{1}{12n}\right) = 1
\end{align*}
Also existiert:
\begin{align*}
    &x = \limmesngegenunendlich a_n \text{ und } c = \limmesngegenunendlich b_n\\
    &\Rightarrow \limmesngegenunendlich \frac{\sqrt{n} \left(\frac{n}{e}\right)^n}{n!} = c > 0
\end{align*}
Gerade Werte einsetzen:
\begin{align*}
    0 < c = \limmesngegenunendlich a_{2n} &= \limmesngegenunendlich \frac{\sqrt{2n}\left(\frac{2n}{e}\right)^{2n}}{(2n)!} = \sqrt{2} \limmesngegenunendlich \underbrace{\frac{n!^22^{2n}}{(2n)! \sqrt{n}}}_{\rightarrow \sqrt{n}} \sqrt{n} \left( \underbrace{\sqrt{\pi} \left(\frac{n}{e}\right)^n}_{a_n}\right)^2\\
    &= \sqrt{2\pi} \limmesngegenunendlich a^2_n = \sqrt{2\pi} \cdot c^2\\
    \Rightarrow c = \frac{1}{\sqrt{2\pi}} &&&\square
\end{align*}
\newcommand{\ndurche}{\left(\frac{n}{e}\right)}
\renewcommand{\summeeinsbisn}{\sum_{i = 1}^{n}}
\begin{align*}
    \log a_n = \log \frac{\sqrt{n} \ndurche^n}{n!} &= \log \sqrt{n} + \log \left(\ndurche^n\right) - \log (n!)\\
    &= \frac{1}{2} \log n + n \log \ndurche - \log \pivoneinsbisn i\\
    &= \frac{1}{2} \log n + n(\log n  - \underbrace{\log e}_{1}) - \sum_{i = 1}^n \log i\\
    &= \frac{1}{2} \log n - n + \sum_{i=1}^{n} \log \frac{n}{i}
\end{align*}

\subsection{Pascalsches Dreieck}
\begin{center}
\begin{tikzpicture}
\node(0) at ( 0, 0) {1};
\node(11) at ( -1, -1) {1};
\node(12) at ( 1, -1) {1};
\node(21) at ( -2, -2) {1};
\node(22) at ( 0, -2) {2};
\node(23) at ( 2, -2) {1};
\node(31) at (-3, -3) {1};
\node(32) at (-1, -3) {3};
\node(33) at (1, -3) {3};
\node(34) at (3, -3) {1};
\node(41) at (-4, -4) {1};
\node(42) at (-2, -4) {4};
\node(43) at (0, -4) {6};
\node(44) at (2, -4) {4};
\node(45) at (4, -4) {1};
\end{tikzpicture}
\end{center}