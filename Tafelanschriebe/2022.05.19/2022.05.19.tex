\subsection{Vortrag \glqq Dinge finden auf Linux\grqq}
Hier wurde von Jonas Zohren (FOSS-AG) ein kurzer Vortrag über Linux gehalten.

\subsection{Hashing}
\underline{Beweis:} $V = \{(s, s') 0 \leq s, s' < p, s \neq s'\}$
\begin{align*}
    f_{s, s'}: (a, b) \mapsto (a \cdot s + b, a \cdot s' + b)\\
   (s, s') \in V; \{ 1, \dots, p-1 \{x\}, 0, ..., q-1\}\rightarrow V
\end{align*}
\begin{center}
\rectangled{Teil 1}\\ ist bijektiv   
\end{center}
\underline{Teil 2:}
\begin{align*}
    U = \{(r, r') \in V: r \mod m = r' \mod m \}
\end{align*}
Dann gilt: $|U| \leq \frac{|v|}{m}$\\
\underline{Denn:} fixiere $r$: Dann produzieren die folgenden Werte den selben Rest $\mod m$\\
\begin{align*}
    r \mod m, m + (r \mod m), 2m + (r \mod m), ..., l \cdot m + (r \mod m)\\
    \text{Insgesamt }\leq \frac{p}{m}, text{eines davon ist genau} \text{mit } l = \lfloor \frac{p}{m}\rfloor
\end{align*}
\begin{align*}
    p(\frac{p}{m} - 1) \leq \frac{|v|}{m} \text{ weil } |V| = p(p -1)
\end{align*}
Aus Teil 1 und 2 folgt der Satz ! \hfill $\square$


\subsection{Graphen}
Keine Anschriften zu diesem Thema. Alle Informationen sind auf den Folien vom 19. Mai 2022.